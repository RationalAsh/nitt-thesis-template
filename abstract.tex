\thispagestyle{plain}
\begin{center}
\textbf{\textbf{\fontsize{16pt}{24pt}\selectfont ABSTRACT}}
\end{center}

\vspace{0.3cm}
\fontsize{12pt}{18pt}\selectfont Removal of colour from industrial wastewater can be achieved by extraction using liquid
emulsion membrane. A dye, named, Crystal Violet (CV) is extracted using water/oil/water
liquid emulsion membrane. An experiment on single dye component is carried out. A stable
emulsion is formed by agitating NaOH solution and an organic solvent (n-hexane) at high
speed. Span 80 (surfactant) is used to stabilize the membrane. Extraction is carried out by
dispersing the emulsion in an external water phase (feed) at lower speed resulting in the
formation of small globules thereby increasing surface area and providing better extraction.
The constituent (dye) to be extracted from the external phase diffuses through the membrane
phase into the internal phase (NaOH solution). Reaction occurs in the internal phase resulting
in the formation of sodium salt of the dye (s). The emulsion can be reused after
demulsification. During extraction, the effect of Span 80, NaOH concentration, n-hexane,
stirring speed and feed concentration have been investigated. The main objective of this study
is to find the optimum operating conditions for the extraction of crystal violet. 

\textit{Keywords}: Emulsion; Internal phase; Extraction; Diffusion; Dye separation

\newpage
